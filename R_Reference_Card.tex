\documentclass{article}
\usepackage[landscape, top=1cm, bottom=1cm, left=1cm, right=1cm]{geometry}
\usepackage{multicol}


\thispagestyle{plain}

\title{EA30 R Reference Card}

\usepackage{Sweave}
\begin{document}
\Sconcordance{concordance:R_Reference_Card.tex:R_Reference_Card.Rnw:%
1 9 1 1 0 73 1}

\maketitle

\begin{multicols}{3}


\section*{Getting help}

\begin{description}
  \item[help.start()] Opens an interactive html page with several help resources
  \item[?'function'] to get help with a specific function, don't include the quotation marks!
  \item[args()] to see the arguments that a function is expecting
\end{description}

\section*{Creating and Importing Data}

\begin{description}
\item[c()]
\item[data.frame()]
\item[read.csv()]
\end{description}

\section*{Statistical Analysis}

\subsection*{Regression}

\begin{description}
\item[lm()] 
\end{description}

\subsection*{ANOVA, t-test, etc}

\begin{description}
\item[t.test()]
\item[aov()]
\end{description}

\subsection*{Tests for Association}

\begin{description}
\item[chisq.test()]
\item[fisher.test()]
\end{description}

\subsection*{Logistic Regression}

\begin{description}
\item[glm(y ~ x, family=binomial(link='logit'))]
\end{description}

\subsection*{Statistical Results}
\begin{description}
\item[summary()]
\item[coef()]
\item[plot()] when you plot a model, you'll get four diagnostic plots that can be used to evaluate the assumptions.
\item[predict()]
\end{description}

\section*{Visual Display}

\begin{description}
\item[plot()] arguments can be in x,y or in a y ~ x formula to create scatter plots
\item[boxplot()] boxplots are used for categorical predictors
\item[abline()]
\item[]

\end{description}

  lots of text
  \ldots
\end{multicols}

\end{document}
