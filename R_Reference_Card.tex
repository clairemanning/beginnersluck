\documentclass{article}\usepackage[]{graphicx}\usepackage[]{color}
%% maxwidth is the original width if it is less than linewidth
%% otherwise use linewidth (to make sure the graphics do not exceed the margin)
\makeatletter
\def\maxwidth{ %
  \ifdim\Gin@nat@width>\linewidth
    \linewidth
  \else
    \Gin@nat@width
  \fi
}
\makeatother

\definecolor{fgcolor}{rgb}{0.345, 0.345, 0.345}
\newcommand{\hlnum}[1]{\textcolor[rgb]{0.686,0.059,0.569}{#1}}%
\newcommand{\hlstr}[1]{\textcolor[rgb]{0.192,0.494,0.8}{#1}}%
\newcommand{\hlcom}[1]{\textcolor[rgb]{0.678,0.584,0.686}{\textit{#1}}}%
\newcommand{\hlopt}[1]{\textcolor[rgb]{0,0,0}{#1}}%
\newcommand{\hlstd}[1]{\textcolor[rgb]{0.345,0.345,0.345}{#1}}%
\newcommand{\hlkwa}[1]{\textcolor[rgb]{0.161,0.373,0.58}{\textbf{#1}}}%
\newcommand{\hlkwb}[1]{\textcolor[rgb]{0.69,0.353,0.396}{#1}}%
\newcommand{\hlkwc}[1]{\textcolor[rgb]{0.333,0.667,0.333}{#1}}%
\newcommand{\hlkwd}[1]{\textcolor[rgb]{0.737,0.353,0.396}{\textbf{#1}}}%
\let\hlipl\hlkwb

\usepackage{framed}
\makeatletter
\newenvironment{kframe}{%
 \def\at@end@of@kframe{}%
 \ifinner\ifhmode%
  \def\at@end@of@kframe{\end{minipage}}%
  \begin{minipage}{\columnwidth}%
 \fi\fi%
 \def\FrameCommand##1{\hskip\@totalleftmargin \hskip-\fboxsep
 \colorbox{shadecolor}{##1}\hskip-\fboxsep
     % There is no \\@totalrightmargin, so:
     \hskip-\linewidth \hskip-\@totalleftmargin \hskip\columnwidth}%
 \MakeFramed {\advance\hsize-\width
   \@totalleftmargin\z@ \linewidth\hsize
   \@setminipage}}%
 {\par\unskip\endMakeFramed%
 \at@end@of@kframe}
\makeatother

\definecolor{shadecolor}{rgb}{.97, .97, .97}
\definecolor{messagecolor}{rgb}{0, 0, 0}
\definecolor{warningcolor}{rgb}{1, 0, 1}
\definecolor{errorcolor}{rgb}{1, 0, 0}
\newenvironment{knitrout}{}{} % an empty environment to be redefined in TeX

\usepackage{alltt}

\usepackage{concmath}
\usepackage[T1]{fontenc}

\usepackage[landscape, top=1cm, bottom=1cm, left=1cm, right=1cm]{geometry}
\usepackage{multicol}

\thispagestyle{plain}

\title{EA30 R Reference Card}
\IfFileExists{upquote.sty}{\usepackage{upquote}}{}
\begin{document}

\maketitle

\begin{multicols}{3}


\section*{Getting help}

\begin{description}
  \item[help.start()] Opens an interactive html page with several help resources
  \item[? `function'] to get help with a specific function, don't include the quotation marks!
  \item[args()] to see the arguments that a function is expecting
\end{description}

\section*{Creating and Importing Data}

\begin{description}
\item[c()] combine or concantenate values, separated by commas
\item[data.frame()] create data frame
\item[read.csv()] read csv file into a data frame
\end{description}

\section*{Statistical Analysis}

\subsection*{Regression}

\begin{description}
\item[lm()] create a linear model (y = f(x) or y $\sim$ x)
\end{description}

\subsection*{ANOVA, t-test, etc}

\begin{description}
\item[t.test()] test two vectors 
\item[aov()] create an object analysis of variance, where y = f(x) or y $\sim$ x
\end{description}

\subsection*{Tests for Association}

\begin{description}
\item[chisq.test()] 
\item[fisher.test()]
\end{description}

\subsection*{Logistic Regression}

\begin{description}
\item[glm(y $\sim$ x, family=binomial(link=`logit'))] 
\end{description}

\subsection*{Statistical Results}
\begin{description}
\item[summary()] coerce object into summary table
\item[coef()] extract coefficients from a model object (e.g. lm)
\item[plot()] when you plot a model, you'll get four diagnostic plots that can be used to evaluate the assumptions.
\item[predict()]
\end{description}

\section*{Visual Display}

\begin{description}
\item[plot()] arguments can be in x,y or in a y $\sim$ x formula to create scatter plots
\item[boxplot()] boxplots are used for categorical predictors
\item[abline()] using values to create vertical, horizonal, or diagonal lines
\item[par()] controls display parameters

\end{description}

  lots of text
  \ldots
\end{multicols}

\end{document}
